\documentclass{article}
\usepackage[T1]{fontenc}
\usepackage[utf8]{inputenc}
\usepackage[margin=1in]{geometry}
\usepackage[bookmarks=true,pdfstartview=FitH,bookmarksopen=true]{hyperref}
\usepackage{bookmark}
\usepackage{url}

\newcommand{\HRule}{\rule{\linewidth}{0.5mm}}
\newcommand{\Hrule}{\rule{\linewidth}{0.3mm}}

\makeatletter% since there's an at-sign (@) in the command name
\renewcommand{\@maketitle}{%
  \parindent=0pt% don't indent paragraphs in the title block
  \centering
  {\Large \bfseries\textsc{\@title}}
  \HRule\par%
  \textit{\@author \hfill \@date}
  \par
}
\makeatother% resets the meaning of the at-sign (@)

\title{Statement of Purpose}
\author{Yongsen MA}
\date{\today}

\begin{document}
  \maketitle% prints the title block

\section{Purpose Statement}
 
To get a better trade-off between data rate and packet delivery, it is essential to make accurate measurement of channel state and link quality. This is more challenging in that multi-configuration in mobile 802.11n not only requires far more samples to acquire sufficient information for all possible channel settings, but also introduces significant complications in channel modeling. Furthermore, channels are more vulnerable to environmental variability and terminal mobility in mobile 802.11n. Therefore, accurate channel measurement and prediction is becoming increasingly important in mobile 802.11n networks.

I have had positive research experiences with my thesis advisor and members of the Center for Intelligent Wireless Networking and Cooperative Control (i-WiN C2)\footnote{\url{http://wicnc.sjtu.edu.cn}}. This, along with interesting coursework, has motivated me to continue into a Doctoral program in wireless networking. Currently, I am working on the energy-efficient issues in 802.11n networks, and I hope to carry on the related researches including MIMO-OFDM systems and mobile applications in the near future. I wish to continue my researches on the development of wireless monitoring and evaluation systems and the implementation of energy-efficient algorithm and cross-layer protocol.

\section{Interests and Publications}

On the physical layer channel state estimation, I written the paper on on-line and dynamic estimation of Rician fading channels in GSM-R networks" It introduces the dynamic EM estimation algorithm to reduce the measurement overhead with guaranteed accuracy and be adaptive to different propagation environments. This paper has been published in WCSP'12, and the journal vision is under review of Springer Wireless Networks. I am also the third author of a paper on dual-antenna based handover scheme for GSM-R network, which is presented in WCSP'12 as well. On the performance measurement of GSM-R networks, I have authored three patents and owned one software copyright.

For the link quality measurement, I mainly work on the packet delivery measurement and prediction in mobile 802.11n networks, In this work, we present the online PDR-RSS modeling framework, which incorporates a novel design by exploiting both packet-level and physical-level metrics, along with the diversity property of multi-configuration simultaneously. This online framework also strikes a balance between the measurement overhead and accuracy. Through a real world implementation on our testbed, The experimental results indicate that it can achieve throughput gains up to 40\% under different MIMO configurations.

In addition to the above papers, I am currently working on the energy-efficient rate adaption in 802.11n networks, seeking the suitable trade-off between data rate and packet delivery under the constraints of energy consumption. The detailed information and documents of my Graduate class projects can be found at my personal home page\footnote{\url{http://yongsen.github.com}}.

\section{Skills and Experiences}

My undergraduate thesis is about the performance evaluation of Zigbee networks based on NS2, which is written in \verb"Tcl" and \verb"Awk"\footnote{Home page: \url{http://yongsen.github.com/ns2_zigbee}, Source code: \url{https://github.com/yongsen/ns2_zigbee}}. It scored 96/100 in the graduation defence and helps me better understand the wireless protocol of IEEE 802.15.4 and network simulator of NS2/NS3. Since that time, I have been working on wireless networking which helps me gain practical experience on programming, simulations and experiments. I built up two performance measurement systems for GSM/GSM-R and 802.11n networks, by which I have gained valuable experience in wireless networking and open source coding. As the instructor of PRP (Participation in Research Program) for undergraduate students, we developed the Um interface monitoring system for GSM/GSM-R networks\footnote{Home page: \url{http://yongsen.github.com/um_monitoring}, Source code: \url{https://github.com/yongsen/um_monitoring}}. The monitor system is written in C\# based on Microsoft .NET Compact Framework and tested along Beijing-Shanghai high-speed railway. Recently, I developed a performance measurement software under Linux system for mobile 802.11n networks\footnote{Home page: \url{http://yongsen.github.com/graded_802.11n}, Source code: \url{https://github.com/yongsen/graded_802.11n}}. The software is written in Linux C based on wireless driver \texttt{ath9k}\footnote{ath9k is a completely open source wireless driver for all Atheros IEEE 802.11n PCI/PCI-Express and AHB WLAN based chipsets, \url{http://linuxwireless.org/en/users/Drivers/ath9k}} and Linux wireless extension.

Additionally, I also participate in research projects including proposals, reports and deliverables. I was responsible for the Key Project of Ministry of Railway on performance measurement in GSM-R networks ,and I also participant in discussions of NSFC projects including dynamic spectrum auction in cognitive radio and demand response in smart grid. Besides, I have the opportunity to participate in conference and journal papers reviewing, mainly including IEEE Infocom, IEEE Globecom, and Springer Wireless Networks. All these experiences equipped me with additional skills to have a better sense of academical and technical problems.

\section{Research Plans}
wireless driver in Linux kernel

mobile applications on Android

energy-efficient spectrum allocation and rate adaption in MIMO-OFDM systems
\end{document}
